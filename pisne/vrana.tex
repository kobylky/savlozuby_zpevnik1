\beginsong{Vrána}[by={Karel Plíhal}]

\beginverse
\chordson
\[D]Byly časy, \[F#mi]kdy i vrána \[Hmi]tak nádherně \[D]zpívala,
\[G]že vedle ní \[D]i ten slavík \[Emi]kníkal jako \[A7]sele.
\chordsoff
\[D]Každé ráno \[F#mi]před rozbřeskem \[Hmi]hodila se \[D]do gala,
\[G]nadechla se, \[D]odkašlala \[Emi]a spustila \[A7]směle:
\endverse

\beginrefrain
\chordson
\[D]Krá, \[F#mi]krá, \[Hmi]krásná je \[D]svoboda, \[G]sladká jak \[D]jahoda \[Emi]na lesní \[A7]skalce.
\[D]Krá, \[F#mi]krá, \[Hmi]krásně je \[D]na světě, \[G]když vítr \[D]nese tě \[A7]tam a zas \[D]tam.
\chordsoff
Krá, krá, krásná je svoboda, hebká jak košile zručnýho tkalce.
Krá, krá, krásně je na světě, když na své „Můžu?“ si odpovím sám.
\endrefrain

\beginverse
\chordsoff
Zatím dole v mraveništi všichni tvrdě makají,
pro matičku, pro královnu pot se z nich jen leje.
Jen co vránu uslyšeli, šeptají si potají,
„nám tu hoří plán a tahle mrcha se nám směje, že prý:“
\endverse
\emptyrefrain

\beginverse
\chordsoff
Královna se rozčílila, teda tohle ničí morálku,
ještě aby mravencům to začlo vrtat šiškou.
To nám ještě scházelo, když chystáme se na válku,
\chordson
a \[G]vydala se \[D]pohovořit \[A7]s loajální \[D]liškou.
\endverse

\beginverse
\chordsoff
Nyní je vám asi jasný, jak to s vránou dopadlo,
ten, kdo umí lítat, ještě nemusí být tabu.
Královně šlo o krk a té lišce zase o žrádlo
\chordson
a \[D]od tý doby \[A]vrány \[D]nejsou svými \[G]pány,
\[D]jenom občas \[H7]v prudkém hněvu \[E7]zazpívaj' si \[A]pro úlevu,
\endverse

\beginspecialverse{Rec.:}
\chordsoff
Když už tak jenom a to navíc většinou bez nároku na honorář:
\chordson
\[D]Krá, krá~…
\endspecialverse
\endsong
