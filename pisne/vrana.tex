\beginsong{Vrána}[by={Karel Plíhal}]

\beginverse
\chordson
\[E]Byly časy, \[G#mi]kdy i vrána \[C#mi]tak nádherně \[E]zpívala,
\[A]že vedle ní \[E]i ten slavík \[F#mi]kníkal jako \[H7]sele.
\chordsoff
\[E]Každé ráno \[G#mi]před rozbřeskem \[C#mi]hodila se \[E]do gala,
\[A]nadechla se, \[E]odkašlala \[F#]a spustila \[H7]směle:
\endverse

\beginrefrain
\chordson
\[E]Krá, \[G#mi]krá, \[C#mi]krásná je \[E]svoboda, \[A]sladká jak \[E]jahoda \[F#mi]na lesní \[H7]skalce.
\[E]Krá, \[G#mi]krá, \[C#mi]krásně je \[E]na světě, \[A]když vítr \[E]nese tě \[H7]tam a zas \[E]tam.
\chordsoff
Krá, krá, krásná je svoboda, hebká jak košile zručnýho tkalce.
Krá, krá, krásně je na světě, když na své „Můžu?“ si odpovím sám.
\endrefrain

\beginverse
\chordsoff
Zatím dole v mraveništi všichni tvrdě makají,
pro matičku, pro královnu pot se z nich jen leje.
Jen co vránu uslyšeli, šeptají si potají,
„nám tu hoří plán a tahle mrcha se nám směje, že prý:“
\endverse
\emptyrefrain

\beginverse
\chordsoff
Královna se rozčílila, teda tohle ničí morálku,
ještě aby mravencům to začlo vrtat šiškou.
To nám ještě scházelo, když chystáme se na válku,
a vydala se pohovořit s loajální liškou.
\endverse

\beginverse
\chordsoff
Nyní je vám asi jasný, jak to s vránou dopadlo,
ten, kdo umí lítat, ještě nemusí být tabu.
Královně šlo o krk a té lišce zase o žrádlo
\chordson
a \[E]od tý doby \[H]vrány \[E]nejsou svými \[A]pány,
\[E]jenom občas \[C#7]v prudkém hněvu \[F#7]zazpívaj' si \[H]pro úlevu,
\endverse

\beginspecialverse{Rec.:}
\chordsoff
Když už tak jenom a to navíc většinou bez nároku na honorář:
\chordson
\[E]Krá, krá~…
\endspecialverse
\endsong
