\beginsong{Dvě spálený srdce}[by={Karel Plíhal}]

\beginverse
\chordson
\[C]{\vphantom{Ž}Tramvají} \[G]{\vphantom{Ž}dvojkou} \[F]{\vphantom{Ž}jezdíval} jsem \[G]{\vphantom{Ž}do} \[C]Ži\[G]{\vphantom{Ž}de}\[F]{\vphantom{Ž}nic}\[G]{\vphantom{Ž},}
z \[C]tak velký \[G]lásky \[F]většinou \[G]nezbyde \[Ami]nic,
z \[F]takový \[C]lásky \[F]jsou kruhy \[C]pod oči\[G]ma
a dvě \[C]spálený \[G]srdce -- \[F]Nagasaki, \[G]Hiroši\[C]ma. \[G] \[F] \[G]
\endverse

\beginverse
\chordsoff
Jsou jistý věci, co bych tesal do kamene,
tam, kde je láska, tam je všechno dovolené,
a tam, kde není, tam mě to nezajímá,
jó, dvě spálený srdce -- Nagasaki, Hirošima.
\endverse

\beginverse
\chordsoff
Já nejsem svatej, ani ty nejsi svatá,
ale jablka z ráje bejvala jedovatá,
jenže hezky jsi hřála, když mi někdy bylo zima,
jó, dvě spálený srdce -- Nagasaki, Hirošima.
\endverse

\beginverse
\chordsoff
= 1.
\endverse

\beginspecialverse{C:}
\chordson
\lrep\ A dvě \[C]spálený \[G]srdce -- \[F]Nagasaki, \[G]Hiroši\[C]ma. \[G] \[F] \[G] \rrep
\endspecialverse
\endsong
