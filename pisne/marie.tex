\beginsong{Marie}[by={Tomáš Klus}]

\beginverse
\chordson
Je \[F]den, tak \[A]pojď Marie ven,
budeme \[B]žít a házet \[C]šutry do oken.
\chordsoff
Je \[F]dva necháme \[A]doma trucovat,
když \[B]nechtějí, nemusí, \[C]nebudem se vnucovat.
Jémine, \[F]všechno \[A]zlý jednou pomine.
Tak \[B]Marie, \[C]co ti je?
\endverse

\beginverse
\chordsoff
Všemocné jsou loutkařovi prsty,
ať jsou tenký nebo tlustý občas přetrhají nit.
A to pak jít a nemít nad sebou svý jistý,
pořád s tváří optimisty listy v žití obracet.
Je to jed, mazat si kolem huby med
a neslyšet, jak se ti bortí svět.
Marie, kdo přežívá nežije, tak ádijé.
\endverse

\beginverse
\chordsoff
Marie, už zase máš tulení sklony,
jako loni slyším kostelní zvony znít
a to mě zabije a to mě zabije a to mě zabije, jistojistě.
\endverse

\beginrefrain
\chordson
Já \[F]mám Marie \[A]rád, když má \[B]moje bytí \[C]spád.
Býti \[F]věčně na cestách\[A] a kránu \[B]spícím plícím
\[C]Život vdechovat, \[F]nechtěj mě milovat,
\[A]nechtěj mě milovat, \[B]nechtěj mě milovat.\[C]
\chordsoff
Já mám Marie rád, když má moje bytí spád,
býti věčně na cestách a kránu spícím plícím
život vdechovat.
\endrefrain

\beginverse
\chordsoff
Copak nemůže být mezi ženou a mužem
přátelství -- kde není nikdo nic dlužen, prostě
jen prosté spříznění duší, aniž by kdokoli cokoli tušil.
Nananana~…
\endverse
\emptyrefrain
\endsong
