\beginsong{Dobrodružství s bohem Panem}[by={Marta Kubišová}]

\beginverse
\chordson
\[Ami]Je půlnoc nádherná, \[G]spí i lucerna, \[F]tys' mě opustil \[E]ospalou,
tu v \[Ami]hloubi zahrady \[G]cítím úklady, s \[F]píšťalou \[E]někdo sem krá\[Ami]čí.
\[C]Hrá náramně, \[G]krásně a na mě \[F]tíha podivná \[E]doléhá.
\[C]Hrá náramně, \[G]zná mě, nezná mě, \[F]něha a \[E]hudba až \[Ami]k pláči.
\endverse

\beginverse
\chordsoff
Pak náhle pomalu skládá píšťalu, krok, a slušně se uklání,
jsem rázem ztracená, co to znamená, odháním strach, a on praví:
Pan jméno mé, mám už renomé, Pan se jmenuju a jsem bůh.
Pan, bůh všech stád, vás má, slečno, rád, jen Pan je pro vás ten pravý.
\endverse

\beginverse
\chordsoff
Ráno, raníčko, ach, má písničko, Pan mi zmizel i s píšťalou,
od Pana, propána, o vše obrána, ospalou najde mě máti.
Hrál a ve tmě krásně podved' mě, kam jsem to dala oči, kam.
Pan, pěkný bůh, já teď nazdařbůh počítám „dal“ a „má dáti“.
\endverse
\endsong