\beginsong{Jacek}[by={Jaromír Nohavica}]

\beginverse
\chordson
\[C]Na druhém břehu řeky \[G]Olše žije Jacek,
\[F]mám k němu stejně blízko \[C]jak on ke mně,
\chordsoff
\[C]máváme na sebe z \[G]říční navigace,
\[F]dva spojenci a dvě \[C]spřátelené země,
jak malí kluci hážem z \[G]břehů žabky,
\[F]kdo vyhraje, má z protěj\[C]šího srandu,
hlavama kroutí česko-\[G]polské babky,
\[F]děláme prostě vlastní \[C]propagandu.
\endverse

\beginrefrain
\chordsoff
Na na na~…
\endrefrain

\beginverse
\chordsoff
Na mostě přátelství se tvoří dlouhé fronty
všelikých věcí za všelikou cenu,
já mám však na to velmi úzké horizonty
a Jacek velmi nenáročnou ženu,
týden co týden z břehů navigace
na sebe řveme: „Chlapče, hlavu vzhůru!“,
jak je to krásné, moci vykašlat se
na celní předpisy a na cenzuru.
\endverse
\emptyrefrain

\beginverse
\chordsoff
Z Piastovské věže na nás mává kníže Měšek
a směje se, až třepe se mu brada,
ve zprávách večer běží horký dnešek,
aspoň se máme s Jackem o co hádat,
on tvrdí svoje, já zas tvrdím svoje
a domluvit se někdy bývá marno,
tak spolu vedem pohraniční boje
a v praxi demonstrujem Solidarnošč.
\endverse
\emptyrefrain

\beginverse
\chordsoff
Na druhém břehu řeky Olzy žije Jacek,
mám k němu stejně blízko jak on ke mně,
máváme na sebe z říční navigace,
dva spojenci a dvě spřátelené země
\lrep\ a voda plyne, plyne, plyne dlouhé věky,
řeka se kroutí jako modrá šňůrka
a my dva hážem kachnám vprostřed řeky
krajíčky chleba o dvou stejných kůrkách. \rrep
\endverse
\emptyrefrain
\endsong
