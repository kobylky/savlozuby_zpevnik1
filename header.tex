\documentclass[12pt,a5paper,twoside,english,czech]{article}

% page geomtery
\usepackage[inner=20mm,outer=5mm,top=7mm,bottom=13mm,footskip=7mm]{geometry}

%%% Velikost a5 stranky je 148 x 210 mm
%\setlength\hoffset{-25.4mm} % 25.4 mm + \hoffset = levy horizointalni okraj
%\setlength\voffset{-25.4mm} % 25.4 mm + \voffset = horn� vertikalni okraj
%\setlength\oddsidemargin{20mm} % leve odsazeni na sude strance
%\setlength\evensidemargin{5mm} % leve odsazeni na liche strance => odsazeni od stredu je: 
%% sirka stranky - \hoffset - 25.4mm - \textwidth - oddsidemargin
%
%%% Zahlavi
%\setlength\topmargin{0mm} % odsazeni zahlavi od horni pozice urcene \voffset
%\setlength\headsep{6mm} % odsazeni textu od zahlavi
%\setlength\headheight{0mm} % vyska zahlavi
%
%%% zapati
%\setlength\footskip{5mm} % odsazeni spodni hranice zapati od hlavniho textu
%
%%% Poznamky u okraje
%\setlength\marginparwidth{5em} % sirka poznamky u kraje
%\setlength\marginparsep{5pt} % odsazeni poznamky u kraje od hlavniho textu
%
%%% velikost oblasti pro psani textu
%\setlength\textwidth{122mm} % sirka textu
%\setlength\textheight{190mm} % vyska textu
%
%%% konec parametru stranky

% used packages
\usepackage[utf8]{inputenc}
\usepackage[pdftex]{graphicx}
\usepackage[english,czech]{babel}
\usepackage{lmodern}% vektorov� fonty Latin Modern, n�stupce p�von�ch Knuthov�ch Computern Modern font�
\usepackage[OT1,T1]{fontenc}% k�dov�n� font�
\usepackage{color}% umo��uje pou��vat barevn� text a tak \textcolor{declared-color}{text}
\usepackage{microtype} %Zapina pouziti mikrotypografie - (protutrusion=true - posunuti pismen na kraji radku,
% expansion=true) - male roztazeni a zuzeni znaku

\usepackage[\songspar,onesongcolumn]{songs}
\usepackage{makeidx}
\makeindex
\usepackage[columns=1,initsep=0pt]{idxlayout}
\usepackage[titles]{tocloft} % customize table of contents

\usepackage{ccicons}

\usepackage[unicode, pdftex,
						colorlinks=false,
						pdfpagemode=UseOutlines,
						pdfdisplaydoctitle=true,
						raiselinks=false,
						pdfhighlight=/N,
						pdfborder={0 0 0},
						pdflang=en,
						pdfstartview=FitH]%,
						{hyperref}

% song settings
\nosongnumbers % switch off song numberse
\setlength{\versenumwidth}{1.3em} % set width of verse numbers

% set style of chords
\renewcommand{\printchord}[1]{\bfseries\sffamily\footnotesize#1}

% set style of sharp adn flat printing
% \renewcommand{\sharpsymbol}{\ensuremath{\#}\,}
% \renewcommand{\flatsymbol}{\ensuremath{\flat}\,}

% size of verse separation
\versesep=10pt plus 5pt minus 3pt
\newlength\tempversesep % for temporary storing versesep

% set separation between chords and text
\renewcommand{\clineparams}{
  \baselineskip=10pt
	\lineskiplimit=0pt
	\lineskip=0pt
}

% switch off chorus bar
\setlength{\cbarwidth}{0pt} % chorus bar width
\setlength{\sbarheight}{0pt} % separator bar height

% defines style in which the authors is showed
\makeatletter
\renewcommand{\showauthors}{ % redefine default not to be boldface
  \setbox\SB@box\hbox{\sfcode`.\@m\songauthors}%
  \ifdim\wd\SB@box>\z@\unhbox\SB@box\par\fi%
}
\makeatother

% defines attributes which are printed in the song header
\renewcommand{\extendprelude}{ % printed header information
  \showauthors
}

% title and header formating
\renewcommand{\makeprelude}{
  \resettitles
	\hypersetup{bookmarksdepth=0}
	\index{\songtitle}
	\addcontentsline{toc}{section}{\songtitle}
	\hypersetup{bookmarksdepth=2}
	{\Large\bfseries\sffamily\songtitle\par\nexttitle\foreachtitle{(\songtitle)\par}}
	{\footnotesize\sffamily\itshape\extendprelude}
}

% margins befor and after song header
\afterpreludeskip=0pt
\beforepostludeskip=0pt

% switch off song footer
\renewcommand{\makepostlude}{}

% set formating of table of contents, it can be used if xindy is not present
%\renewcommand{\cftsecleader}{\cftdotfill{\cftsubsecdotsep}}
%\renewcommand{\cftsecfont}{\normalfont}
%\renewcommand{\cftsecpagefont}{\normalfont}
%\renewcommand{\cftdotsep}{1}
%\renewcommand{\cftbeforesecskip}{0pt}
%% \cftsetindents{subsection}{0em}{0em}

\newcommand\lettergroup[1]{} % switches off letters generated at the beginning of each
% lettergroup by xindy

% custom macros

 % implement refrain as verse with leading R: 
\makeatletter
\newif\ifvspecial
\providecommand\vverseleadingchars{}
\newenvironment{refrain}
{
  \vspecialtrue\renewcommand\vverseleadingchars{R:}
	\beginverse
}{%
	  \endverse%
		\vspecialfalse
	  \addtocounter{versenum}{-1}% decrease increased counter
}
\newcommand\beginrefrain{%
  \vspecialtrue\renewcommand\vverseleadingchars{R:}\beginverse%
}
\makeatother

 % implement special as verse with leading user defined parameter: 
\makeatletter
\newif\ifvspecial
\newenvironment{specialverse}[1]
{
  \vspecialtrue\renewcommand\vverseleadingchars{#1}
	\beginverse
}{%
  \endverse%
	\vspecialfalse%
  \addtocounter{versenum}{-1}% decrease increased counter
}
\newcommand\beginspecialverse[1]{%
  \vspecialtrue\renewcommand\vverseleadingchars{#1}
  \beginverse%
}
\makeatother

% redefine \printversenum to print special character for refrains etc.
\renewcommand{\printversenum}[1]{\ifvspecial {\lyricfont\vverseleadingchars} \else {\lyricfont#1.} \fi}

% define emptyspecialverse so that it is tightly attached to the preceding verse (without any margin)
\makeatletter
\newcommand\emptyspecialverse[1]{%
  \SB@inversetrue%
	\def\SB@closeall{\endsong}
	\global\SB@stanzatrue%
	\setbox\SB@box\vbox\bgroup\begingroup%
	  \setbox\SB@box\hbox{#1}%
		\ifdim\wd\SB@box<\versenumwidth%
		  \setbox\SB@box%
			\hbox to\versenumwidth{\unhbox\SB@box\hfil}%
		\fi%
		\placeversenum\SB@box%
	\endgroup\egroup
	\SB@putbox\unvbox\SB@box
	\SB@inversefalse
	\SB@prevversetrue
}
\makeatother

% define \emptyspecialrefrain to be tightly attached to the preceding verse
\newcommand\emptyrefrain{\emptyspecialverse{R:}}

% redefine default \lrep a \rrep not to extend to the chords line
\makeatletter
\renewcommand\lrep{%
  \ifchorded%
    \SB@dimen.5\baselineskip%
	\else%
	  \SB@dimen\baselineskip%
	\fi%
	\advance\SB@dimen-2\p@%
	\ifvmode\leavevmode\fi% leave vertical mode to enable raise
	\raise-.5\p@\hbox{%
	  \vrule\@width1.5\p@\@height\SB@dimen\@depth\p@%
	  \kern1.5\p@%
	  \vrule\@width.5\p@\@height\SB@dimen\@depth\p@%
	}%
	\raise1.25\p@\hbox{:}%
}
\renewcommand\rrep{%
  \raise1.25\p@\hbox{:}%
  \ifchorded%
    \SB@dimen.5\baselineskip%
	\else%
	  \SB@dimen\baselineskip%
	\fi%
	\advance\SB@dimen-2\p@%
	\raise-.5\p@\hbox{%
	  \vrule\@width.5\p@\@height\SB@dimen\@depth\p@%
	  \kern1.5\p@%
	  \vrule\@width1.5\p@\@height\SB@dimen\@depth\p@%
	}%
}
\makeatother

% redefine \rep not to extend parenthesees under line
\makeatletter
\renewcommand\rep[1]{%
  \ifvmode\leavevmode\fi% leave vertical mode to enable raise
  \raise.5\p@\hbox{(}%
	#1%
	\raise.25ex\hbox{%
		\fontencoding{OMS}\fontfamily{cmsy}\selectfont\char\tw@%
  }%
	\raise.5\p@\hbox{)}%
}
\makeatother

% switch off all chords if chords are not used
\def\chordedtoken{chorded}
\ifx\chordedtoken\songspar
\else
  \renewcommand{\chordson}{\chordsoff}
\fi

% redefines \colbotglue, which is used by the \brk command
% when the page is broken to fill the rest of the page by
% the blank space instead of stretch the song on the whole
% page in lyric mode (in chorded mode, this behavior is by
% default).
\makeatletter
\iflyric\def\colbotglue{\z@\@plus.5\textheight}\fi
\makeatother

\newcommand\mytocline[2]{\hyperlink{page.#2}{#1\dotfill#2}} % hyperref whole toc line to page

% define macro, which clears to the page divisible by four. It is used for the last (cover) page
% and you have to be carefull with using it, because if you reset counter somewhere, you have
% to reflect page numbering shift
\makeatletter
\newcommand\cleartofourthpage[1]{%
  \clearpage%
  \count@=\numexpr\c@page+#1\relax% adds to counter to reflect the page numbering shift
	\@tempcnta=\count@%
	% compute modulo
	\divide\@tempcnta by 4%
	\multiply\@tempcnta by 4%
	\count@=\numexpr\count@-\@tempcnta\relax%
	% clear pages
	\loop\ifnum\count@<4%
		\ifnum\count@=3% no number on the page before last page
			\pagestyle{empty}%
	  \fi%
		\null\clearpage%
		\advance\count@\@ne%
	\repeat%
}
\makeatother
